\subsection{OpenVPN IP Address Allocation in Server Mode}

The allocation of IP addresses to OpenVPN clients in server mode varies depending on the chosen configuration: \texttt{net30}, \texttt{subnet}, or \texttt{p2p} (point-to-point). The following details the IP address assignment for each mode.

\subsubsection{net30 (Default Mode up to OpenVPN 2.4)}
\begin{itemize}
    \item \textbf{Description}: Each client is assigned a dedicated /30 subnet, which reserves 4 IP addresses (one for the network, one for broadcast, and two usable addresses: one for the client and one for the server). Only Windows clients will set it up as a /30 subnet, non Windows clients will set it up as a true Point-to-Point config, treating it as a p2p topology.
\item \textbf{IP Assignment}:
\begin{itemize}
        \item The server gets the first usable IP address in the /30 subnet.
        \item The client gets the second usable IP address.
\end{itemize}
\item \textbf{Example}:
\begin{itemize}
\item Subnet: \texttt{10.8.0.4/30}
\item Server: \texttt{10.8.0.5}
\item Client: \texttt{10.8.0.6}
\item Broadcast: \texttt{10.8.0.7}
\end{itemize}
\item \textbf{Implications}: This configuration wastes half of the IP addresses in each subnet and requires additional routing to reach clients due to the isolated subnets.
\end{itemize}

\subsubsection{subnet (Default Mode since OpenVPN 2.4)}
\begin{itemize}
    \item \textbf{Description}: All clients share the same subnet as the server. This mode uses a single subnet, making it more efficient than \texttt{net30}.
\item \textbf{IP Assignment}:
\begin{itemize}
        \item The server gets the first IP address in the defined subnet.
        \item Clients receive IPs from the pool within the subnet.
\end{itemize}
\item \textbf{Example}:
\begin{itemize}
\item Subnet: \texttt{10.8.0.0/24}
\item Server: \texttt{10.8.0.1}
\item Client 1: \texttt{10.8.0.2}
\item Client 2: \texttt{10.8.0.3}
\item Broadcast: \texttt{10.8.0.255}
\end{itemize}
\item \textbf{Benefits}: This mode is well-suited for scenarios requiring multiple clients to interact as if on the same Local Area Network (LAN).
\end{itemize}

\subsubsection{p2p (Point-to-Point)}
\begin{itemize}
\item \textbf{Description}: This configuration establishes a direct connection between the server and a single client, bypassing the use of a shared subnet. IP address assignment is typically manual or explicitly pre-configured. p2p is not usable on Windows systems.
\item \textbf{IP Assignment}:
\begin{itemize}
\item Both the server and the client are assigned specific IP addresses from a predetermined range.
\end{itemize}
\item \textbf{Example}:
\begin{itemize}
\item Server: \texttt{10.8.0.1}
\item Client: \texttt{10.8.0.2}
\end{itemize}
\item \textbf{Common Use Case}: This mode is frequently employed for site-to-site Virtual Private Networks (VPNs).
\end{itemize}

\subsubsection{Consequences of Using \texttt{net30}}
\begin{itemize}
\item \textbf{Inefficient IP Usage}: The \texttt{net30} configuration allocates a /30 subnet to each client, which includes four IP addresses. Out of these, only two are actively utilized (one for the server and one for the client), leading to a 50\% utilization rate and significant wastage.
\item \textbf{Increased Management Overhead}: Managing numerous individual /30 subnets can introduce complexity, especially when the requirement is for clients to communicate with each other. This mode is optimized for isolated connections rather than collaborative network environments.
\item \textbf{Compatibility}: The only reason to use \texttt{net30} is for supporting Windows clients before 2.0.9, which is incredibly rare nowadays. 
\item \textbf{No Broadcast Domain}: There's no shared broadcast domain among clients. Broadcast traffic is limited to each client-server pair.
\end{itemize}

\subsubsection{Enabling Inter-Client Communication}
\begin{itemize}
    \item \textbf{Subnet Mode}: In \texttt{subnet} mode, clients are assigned IP addresses from the same subnet as the server. Since they are within the same network, clients can communicate directly with each other using their assigned IP addresses. 
    \item \textbf{Server Routing}: In \texttt{net30} or \texttt{p2p} modes, clients are typically isolated from one another by default. To enable client-to-client communication, the server can be configured to route traffic between clients.
    \item \textbf{Client-to-Client Directive}: In the server configuration file, you can enable the \texttt{client-to-client} directive. This directive allows OpenVPN clients to communicate directly with each other, even when using \texttt{net30} or \texttt{p2p} mode. 
    \item \textbf{Firewall Rules}: If communication is not working, ensure that the server's firewall is configured to allow traffic between clients.
\end{itemize}

\subsubsection{Summary}
\[
\begin{array}{|c|c|c|c|}
\hline
\textbf{Mode} & \textbf{Server IP} & \textbf{Client IP} & \textbf{Key Features} \\
\hline
\texttt{net30} & \text{First usable IP} & \text{Second usable IP} & \text{Separate /30 subnets for each client, IP wastage} \\
\hline
\texttt{subnet} & \text{First IP in subnet} & \text{IPs from subnet pool} & \text{Shared subnet for all clients} \\
\hline
\texttt{p2p} & \text{Manually defined} & \text{Manually defined} & \text{Direct point-to-point link, no Windows support} \\
\hline
\end{array}
\]

\subsection{Alternative Authentication Methods in OpenVPN}

OpenVPN offers several methods for authenticating clients beyond the use of certificates. These alternatives provide increased security and flexibility. Common methods include:

\begin{itemize}
\item \textbf{Username and Password Authentication}: OpenVPN can be configured to authenticate clients using a username and password combination. This can be used alongside certificates or as a standalone method, often integrated with external authentication systems.
\item \textbf{Two-Factor Authentication (2FA)}: Enhance security by requiring a second authentication factor, such as a one-time password (OTP) from apps like Google Authenticator, in addition to a username and password.
\item \textbf{Token-based Authentication}: Utilize time-based one-time passwords (TOTP) or push notification services (e.g., Duo, Authy) for a more robust authentication process.
\item \textbf{Integration with External Authentication Backends}: OpenVPN can integrate with directory services like LDAP (Lightweight Directory Access Protocol), RADIUS (Remote Authentication Dial-In User Service), and Active Directory for centralized user management.
\end{itemize}

OpenVPN version 2.0 and later supports client authentication using both username/password credentials and certificates. The OpenVPN server can receive and validate the username and password transmitted securely over the TLS channel.

\paragraph{Client Configuration:}
To enable username/password authentication, add the \texttt{auth-user-pass} directive to the client configuration file. This prompts the user for their credentials upon connection attempt.

\paragraph{Server Configuration:}
The OpenVPN server needs an authentication plugin to handle username/password validation. This plugin can be a script or a compiled module that the server executes upon client connection attempts. The plugin verifies the provided credentials and informs the server of the authentication outcome.

\paragraph{Utilizing Script Plugins:}
The \texttt{auth-user-pass-verify} directive in the server configuration specifies a script for authentication. For example:
\begin{minted}{text}
auth-user-pass-verify /etc/openvpn/auth-pam.pl via-file
\end{minted}
This example uses a Perl script, \texttt{auth-pam.pl}, to authenticate against the system's Pluggable Authentication Modules (PAM). While this script is included for demonstration, using a compiled plugin like \texttt{openvpn-auth-pam} is recommended for production due to enhanced security and performance.

\paragraph{Employing Shared Object or DLL Plugins:}
Compiled plugins offer a more secure and efficient authentication method. On Linux systems, the \texttt{openvpn-auth-pam} plugin integrates with the PAM system. Add the following to the server configuration:
\begin{minted}{text}
plugin /usr/lib/openvpn/openvpn-auth-pam.so login
\end{minted}
This approach is favored for production environments due to its use of a split-privilege model, which improves security, and the faster execution speed of compiled code.

\paragraph{Bypassing Client Certificates:}
Although generally discouraged for security reasons, OpenVPN can be configured to authenticate clients solely with usernames and passwords. The \texttt{client-cert-not-required} directive on the server removes the necessity for client certificates. The \texttt{username-as-common-name} directive can then be used to treat the provided username as the Common Name (CN) for credential indexing.

It's crucial to understand that while client certificates can be bypassed, the server's certificate remains essential for clients to verify the server's identity. Therefore, the \texttt{ca} directive should still be present in the client configuration.

\paragraph{OpenVPN Access Server Authentication Options:}
The OpenVPN Access Server provides a user-friendly interface for managing various authentication methods, including:

\begin{itemize}
\item \textbf{Local Authentication}: Authenticate users against an internal database managed through the Access Server's web interface.
\item \textbf{LDAP Authentication}: Integrate with LDAP directories like Active Directory for centralized user management.
\item \textbf{RADIUS Authentication}: Authenticate users against RADIUS servers for centralized authentication services.
\item \textbf{SAML Authentication}: Enable Single Sign-On (SSO) through integration with SAML identity providers such as Okta or Google Workspace.
\item \textbf{PAM Authentication}: Leverage the Pluggable Authentication Modules (PAM) for integration with a variety of authentication systems.
\end{itemize}

Configuration of these methods is straightforward via the administrative web interface or command-line tools using the \texttt{auth.module.type} setting. The Access Server also supports multi-factor authentication (MFA), allowing the combination of different methods for enhanced security. For instance, users might be required to provide a password and a one-time code from Google Authenticator. This flexibility makes the Access Server suitable for diverse enterprise security requirements.

\subsection{Data Compression in OpenVPN}

OpenVPN includes support for data compression, which can reduce the amount of data transmitted over the network. OpenVPN can be configured to use the \texttt{LZO} compression algorithm. This algorithm is selected for its balance between compression efficiency and processing speed, making it suitable for real-time VPN connections.

\subsubsection{Quantifying Compression Savings}

Enabling compression in OpenVPN results in data being compressed before transmission and then decompressed upon arrival. The degree of compression achieved depends significantly on the nature of the data being transmitted. Typical compression ratios can range from 2:1 to 3:1. This implies a potential reduction in data volume of 50\% to 67\%. For instance, a 1 MB file could be reduced to approximately 500 KB to 333 KB. However, it's important to consider that data already compressed, such as media files (images, videos) or certain web content formats, will experience minimal or no additional compression.

\subsubsection{The Risk of Combining Compression with TLS}

While compression offers bandwidth savings, its use in conjunction with Transport Layer Security (TLS) is generally discouraged due to security vulnerabilities. The primary concern is the \textit{CRIME} attack (Compression Ratio Info-leak Made Easy). This attack exploits the characteristics of compression algorithms to infer information about the content of encrypted data. An attacker can analyze the size of compressed data streams to deduce information about the plaintext, effectively compromising the encryption.

Further compounding this issue is the \textit{VORACLE} vulnerability (VPN Oracle Compression Leak), specifically targeting OpenVPN implementations. VORACLE allows attackers to leverage compressed traffic to gain insights into the plaintext data by exploiting how OpenVPN manages compression.

Due to these significant security risks, OpenVPN has disabled compression by default when TLS is in use. It is a recommended security practice to avoid using compression with VPNs that rely on robust encryption protocols like TLS.

\subsection{Client-Specific Parameter Configuration in OpenVPN}

OpenVPN offers several ways to assign parameters unique to individual clients. This allows for granular control over client connections, enabling configurations like directing clients to specific subnets or restricting access to particular servers.

\subsubsection{Individual Client Configuration Files}

One common method involves creating individual configuration files for each client. These files, stored on the server, can specify client-specific settings, such as static IP addresses, routing configurations, and access control rules. These files are usually linked to clients via their certificates or unique identifiers.

For example, to assign a specific IP address and subnet mask to a client, the following directive can be added to their individual configuration file:
\begin{minted}{bash}
ifconfig-push 10.8.0.10 255.255.255.0
\end{minted}
This assigns the IP address \texttt{10.8.0.10} and the subnet mask \texttt{255.255.255.0} to that specific client.

\subsubsection{Utilizing the \texttt{ccd} Directory}

Another approach uses the \texttt{ccd} (Client Configuration Directory). When the OpenVPN server is configured with the \texttt{client-config-dir} directive, it looks for a directory containing configuration files named after the client certificate's Common Name (CN). For a client with the CN \texttt{client1}, their configuration file would be named \texttt{ccd/client1}.

This file can contain directives to assign static IP addresses or push specific routes:
\begin{minted}{bash}
# Assign a static IP address
ifconfig-push 10.8.0.10 255.255.255.0

# Push specific routes to the client
push "route 192.168.1.0 255.255.255.0"
\end{minted}
This allows for directing clients to specific network segments or resources.

\subsubsection{Restricting Server Access}

Limiting client access to specific servers or resources can be achieved through Access Control Lists (ACLs) or firewall rules implemented on the OpenVPN server. By configuring the server's routing and firewall, traffic from specific clients can be restricted to only reach designated destinations.

For instance, using \texttt{iptables} on a Linux-based server, rules can be set up to permit a client to access only a specific subnet:
\begin{minted}{bash}
# Restrict client access to specific subnet
iptables -A FORWARD -i tun0 -s 10.8.0.10 -d 192.168.2.0/24 -j ACCEPT
iptables -A FORWARD -i tun0 -s 10.8.0.10 -d 0.0.0.0/0 -j REJECT
\end{minted}
This example allows the client with IP address \texttt{10.8.0.10} to access only the \texttt{192.168.2.0/24} subnet, blocking all other outgoing traffic.

\subsection{Certificate Revocation List (CRL) Implementation in OpenVPN}

In OpenVPN, managing certificate revocation is crucial for maintaining security. A Certificate Revocation List (CRL) allows administrators to revoke certificates that are no longer valid or have been compromised. This ensures that clients with revoked certificates cannot access the VPN, even if they still possess valid, but invalidated, credentials.

\subsubsection{Generating a CRL for a Client Certificate}

The creation and management of CRLs are typically performed using the OpenSSL toolkit. The process to create a CRL involves the following steps:

\begin{enumerate}
\item \textbf{Revoking the Certificate}: The first step is to formally revoke the client certificate using the OpenSSL command:
\begin{minted}{bash}
openssl ca -revoke /path/to/client-cert.pem -keyfile /path/to/ca.key -cert /path/to/ca.crt
\end{minted}
This command marks the specified certificate as revoked within the CA's database.

\item \textbf{Generating the CRL File}: After revoking the certificate, the CRL file needs to be generated. This file contains the list of all revoked certificates. Use the following command:
\begin{minted}{bash}
openssl ca -gencrl -out /path/to/crl.pem -keyfile /path/to/ca.key -cert /path/to/ca.crt
\end{minted}
This command creates the updated CRL file at the specified path.

\item \textbf{CRL Publication}: The generated CRL needs to be accessible to the OpenVPN server. This often involves placing the \texttt{crl.pem} file in a location readable by the OpenVPN service.

\end{enumerate}

\subsubsection{Verifying CRL Effectiveness}

To ensure the CRL is functioning correctly, the following steps can be taken:

\begin{enumerate}
\item \textbf{Configuring OpenVPN for CRL Verification}: The OpenVPN server needs to be configured to check the CRL. This is done by adding the \texttt{crl-verify} directive to the server configuration file:
\begin{minted}{bash}
crl-verify /path/to/crl.pem
\end{minted}
This directive instructs the OpenVPN server to consult the specified CRL during client connection attempts.

\item \textbf{Testing with a Revoked Certificate}: Attempt to connect to the OpenVPN server using the client certificate that was added to the CRL. The connection should be refused by the server.

\item \textbf{Log Verification}: Examine the OpenVPN server logs to confirm that the CRL was checked and the connection was blocked due to the revoked certificate. The logs should indicate that the client's certificate was found in the CRL.

\end{enumerate}

\subsubsection{Alternative Methods for Distributing Revoked Certificate Information}

Besides CRLs, another significant method for communicating certificate revocation status is the Online Certificate Status Protocol (OCSP), which provides real-time status checks.

\subsubsection{Utilizing OCSP with OpenVPN}

OCSP offers a more immediate way to check if a certificate is still valid compared to the periodic updates of CRLs. While OpenVPN doesn't natively support OCSP, it can be implemented using external scripts. A notable example is the \texttt{OCSP\_check.sh} script available in the OpenVPN GitHub repository:

\url{https://raw.githubusercontent.com/OpenVPN/openvpn/refs/heads/master/contrib/OCSP_check/OCSP_check.sh}

This script allows the OpenVPN server to query an OCSP responder to ascertain the revocation status of a client certificate in real time during the connection process. This approach enhances security by providing up-to-the-minute revocation information.

\subsubsection{CRL vs. OCSP: A Comparison}

\begin{enumerate}
\item \textbf{Certificate Revocation List (CRL)}:
\begin{itemize}
\item CRLs are distributed periodically and require clients (or in this case, the server) to download the entire list.
\item Revocation information is only as current as the last CRL update.
\item Simpler to implement in a basic setup but less real-time.
\end{itemize}

\item \textbf{Online Certificate Status Protocol (OCSP)}:
\begin{itemize}
    \item Provides real-time, on-demand checking of a certificate's status.
    \item Requires an OCSP responder infrastructure.
    \item Offers more timely revocation information, enhancing security in dynamic environments.
\end{itemize}
\end{enumerate}

While CRLs are more straightforward to configure within OpenVPN, OCSP offers a more robust solution for immediate revocation status verification, albeit with additional setup requirements.
