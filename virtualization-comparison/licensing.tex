
\section{Licensing}

\subsection{Comparison}

\paragraph{Proxmox} 
is \textbf{open-source} and offers all features for free by default for private and/or commercial use. Additionally, Proxmox offers 4 simple subscription-based models for enterprises which offer technical support and access to the \enquote{Proxmox Enterprise Repository}, which provides reliable updates and security patches. The available subscription models are:
\begin{itemize}
	\item \textbf{Community}:\num{ 110}€/year \& CPU-Socket
	\item \textbf{Basic} \num{340}€/year \& CPU-Socket
	\item \textbf{Standard} \num{510}€/year \& CPU-Socket
	\item \textbf{Premium} \num{1020}€/year \& CPU-Socket
\end{itemize}
All of them offer access to the enterprise repository, but they differ in the quality and quantity of their technical support.\newline
An example setup consisting of 3 hosts and a 32-core CPU each will cost from \num{330}€/year to \num{3060}€/year depending on the amount of support wanted. (or nothing on the free edition)


% https://community.veeam.com/blogs-and-podcasts-57/decoding-the-new-broadcom-vmware-vsphere-licensing-packages-for-small-deployments-6398
\paragraph{VMware} requires more complex licensing.\newline
ESXi hosts are licensed with vSphere licenses. There is one main license model and a few older models which are no longer sold but are still supported. The current licensing model is per core licensing with a minimum of 16 cores per CPU. All licenses are sold in different editions:\newline 
\textit{The following prices are based on the MSRP when using a 3-year subscription. VMware offers 1, 3 and 5 year subscriptions with varying prices.}

\subparagraph{vSphere Essentials Plus Kit}
is sold in 96 core license packs and includes vCenter Essentials for up to 3 hosts with a price of \num{35}\$/core/year which totals to \num{3360}\$/year or \num{10080}\$/3-years.

\subparagraph{vSphere Standard} includes vCenter Server Standard and is priced at \num{50}\$/core/year. It does not have a upper limit on hosts or cores. A license with 96 cores would cost \num{14400}\$/3-years 

\subparagraph{vSphere Foundation} includes vSphere Enterprise Plus, vCenter Server Standard, Tanzu Kubernetes Grid, Aria Suite Standard and available Add-On's. Additionally it includes 100GiB of vSAN Enterprise per-core. The price is \num{135}\$/core/year. 
A license with 96 cores would cost \num{38880}\$/3-years.


\subparagraph{Other models,} which are no longer being sold, include: 
\begin{itemize}
	\item per-CPU licensing with a maximum of 32 cores per license. For CPUs with more than 32 cores multiple licenses have to be acquired
	\item pre-VM licensing
	\item vSphere+ capacity based licensing
\end{itemize}

\subparagraph{Evaluation} is possible with a 60-day trial license.
It starts as a free 60-day trial and can be converted to a full license by purchasing a license key. The license key is entered into the vSphere Client and the evaluation license is converted to a full license.

\subsection{Conclusion}

Proxmox's licensing solely provides additional support and updates, while the software itself is free and open-source. 
VMware is proprietary and not free. The licensing is more complex and expensive, but it offers more features and support. 
