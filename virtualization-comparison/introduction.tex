This document will compare the two hypervisors Proxmox and vSphere.
\\\\
Proxmox is an open-source hypervisor based on Debian. Proxmox is mostly used for home labs or small businesses. It is free to use, but you can buy a subscription to get support and additional features.
\\\\
vSphere is a proprietary hypervisor developed by VMWare. It is used in enterprise environments and is not free.

\subsection{Understanding vSphere}

vSphere has a lot of different components, with different names. Understanding the naming will be crucial for the rest of the document.
\\\\
\begin{itemize}
	\item vSphere: The entire virtualization platform
	\item ESXi: The hypervisor running on the physical hardware
	\item vCenter: The service for managing multiple ESXi hosts
\end{itemize}

\subsection{Installation}

We tried to install both hypervisors to test them and make images for this document, but we ran into issues with vSphere.
\\\\
Both the ESXi 7.0 and 8.0 ios provided by out teacher did not run under Hyper-V, always getting stuck on "Relocating modules and starting up the kernel".

With KVM the installation worked, but we were not able to connect to the web interface, because of some problem with ESXi and the virtualized network interface.

Finally the VMWare Workstation VM provided by out teacher did work, however VMWare Workstation needs Hyper-V disable and memory integrity turned off on a Windows 11 host to use nested virtualization. Since we use Hyper-V for Proxmox and turning off memory integrity is a security risk we choose not to do this. Without nested virtualization we were not able to run a VM inside ESXi.