\section{Management}
This section will describe the management possibilities of Proxmox and vSphere. 

\subsection{Proxmox}
Proxmox has a web interface for managing the hypervisor. The web interface is very user-friendly and easy to use. You can manage all your VMs, containers, and storage from the web interface. You can also manage your network settings and firewall from the web interface. Proxmox also has a command-line interface for managing the hypervisor. The command-line interface is very powerful and allows you to do everything you can do in the web interface and more. Proxmox also offers a REST-like API for managing the hypervisor. The API is very powerful, has API clients in multiple programming languages and allows you to automate many tasks and integrate with different systems. 
\paragraph{Cluster Management} 
Proxmox has a built-in cluster management feature that allows you to manage multiple Proxmox nodes from a single web interface. You can create a cluster by adding multiple Proxmox nodes to the cluster. Once you have created a cluster, you can manage all your nodes from a single web interface. You can create VMs and containers on any node in the cluster, migrate VMs and containers between nodes, and manage storage and network settings across all nodes in the cluster. Proxmox also has a built-in high availability feature that allows you to configure VMs and containers to automatically migrate to another node in the cluster if a node fails. 

\subsection{vSphere}
