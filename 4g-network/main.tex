%%%%%%%%%%%%%%%%%%%%%%%%%%%%%%%%%%%%%%%%%
% Wenneker Assignment
% LaTeX Template
% Version 2.0 (12/1/2019)
%
% This template originates from:
% http://www.LaTeXTemplates.com
%
% Authors:
% Vel (vel@LaTeXTemplates.com)
% Frits Wenneker
%
% License:
% CC BY-NC-SA 3.0 (http://creativecommons.org/licenses/by-nc-sa/3.0/)
% 
%%%%%%%%%%%%%%%%%%%%%%%%%%%%%%%%%%%%%%%%%

%----------------------------------------------------------------------------------------
%	PACKAGES AND OTHER DOCUMENT CONFIGURATIONS
%----------------------------------------------------------------------------------------

\documentclass[11pt]{scrartcl} % Font size

\input{structure.tex} % Include the file specifying the document structure and custom commands

%----------------------------------------------------------------------------------------
%	TITLE SECTION
%----------------------------------------------------------------------------------------

\title{	
	\normalfont\normalsize
	\begin{center}
		\begin{minipage}[c]{0.2\textwidth}
			\textsc{\Large SZ-Ybbs}
		\end{minipage}%
		\begin{minipage}[c]{0.1\textwidth}
			\includegraphics[width=\textwidth]{LogoITHTL_white.pdf}
		\end{minipage}
	\end{center}
	\vspace{10pt} % Whitespace
	\rule{\linewidth}{0.5pt}\\ % Thin top horizontal rule
	\vspace{20pt} % Whitespace
	{\huge 4G}\\ % The assignment title
	\vspace{12pt} % Whitespace
	\rule{\linewidth}{2pt}\\ % Thick bottom horizontal rule
	\vspace{12pt} % Whitespace
}

\author{\LARGE Erber Jakob \and \LARGE Freunberger Raphael} % Your name

\date{\normalsize\today} % Today's date (\today) or a custom date

\begin{document}

\maketitle % Print the title

\tableofcontents
\clearpage

\section{Introduction}
This paper explores key technologies that enable the efficient operation of 4G wireless networks, with a particular focus on Orthogonal Frequency Division Multiple Access (OFDMA) and Multiple-Input Multiple-Output (MIMO) systems. The first part of the paper examines the fundamental requirements of a 4G network, followed by an in-depth analysis of OFDMA, a key multiple access technique used in modern communication systems. OFDMA's advantages over traditional frequency division multiplexing (FDM) and its ability to efficiently allocate subcarriers among multiple users are highlighted. The paper then delves into MIMO technology, which utilizes multiple antennas to improve data rates and reliability, and its various configurations and applications within 4G networks. Together, OFDMA and MIMO represent the backbone of 4G wireless communication, enabling high-speed, robust, and scalable mobile connectivity.

\section{Requirements of a 4G Network}
A 4th generation (4G) mobile radio system was designed to fulfill a set of stringent functional and performance requirements, significantly surpassing the capabilities of its 3G predecessors. These requirements were driven by the increasing demand for high-speed data, multimedia services, and seamless mobility. The International Telecommunication Union-Radio communications sector (ITU-R) played a crucial role in defining these requirements.

\subsection{Performance Requirements}

The ITU-R formally defined the performance requirements through the International Mobile Telecommunications Advanced (IMT-Advanced) specification. The requirements were set quite ambitious for the 2008, with peak data rates of 100 megabits per second (Mbit/s) for high mobility communication and 1 gigabit per second (Gbit/s) for low mobility communication. These peak data rates were intended to support a wide range of data-intensive applications, including high-definition video streaming, online gaming, and cloud services.

These requirements could not be met by the first verson of LTE, and ITU-R later droped the ambitious performance requirements. Calling LTE anything other than 4G would have caused significant confusion, as it is not compatible with 3G and a major improvement.

Other major performance requirements outlined in the IMT-Advanced specification include:

\begin{itemize}
    \item \textbf{Spectral Efficiency:} 4G aims for significantly improved spectral efficiency compared to 3G. As the spectrum avalible for civilian use is very limited improved spectral efficiency was required to seve more useres with increasing data rates. Technologies like OFDMA and MIMO were parts of the IMT-Advanced requirements for achieving this for achieving better spectral efficiency. The IMT-Advanced specification targeted peak spectral efficiencies of 15 bps/Hz for the downlink and 6.75 bps/Hz for the uplink.
    \item \textbf{Latency:} Reduced latency is a key requirement for 4G. The latency for establishing a connection with a cell tower should be less than 100ms and the round-trip time for data packets should be less than 10ms.
    \item \textbf{Mobility:} The system is required to support high mobility. The requirements specify that the system must support travel speeds of up to 350km/h.
\end{itemize}

\subsection{Functional Requirements}

Functional requirements define the core capabilities and features that a 4G system must possess. Key functional requirements include:

\begin{itemize}
    \item \textbf{Seamless Mobility and Handover:} 4G systems must provide seamless mobility across different cells and even between different link technologies.
    \item \textbf{Quality of Service (QoS) Differentiation:} 4G networks must be capable of differentiating between different types of traffic and providing varying levels of QoS.
    \item \textbf{Security:} Different combinations of integrety and encryption must be supported to ensure the security of user data and privacy.
\end{itemize}

\section{OFDMA - Erber Jakob}
Orthogonal frequency-division multiplexing (OFDM) and the multi user variant OFDMA are in nearly all modern wireless communication systems, including 4G, 5G, and Wi-Fi 6. OFDM is a modulation technique that divides the available spectrum into multiple orthogonal subcarriers. This allows for high spectral efficiency and robustness against frequency-selective fading. OFDMA is a multiple access scheme that allows multiple users to share the same frequency band by assigning different subcarriers to different users.

But lets start with Frequency Division Multiplexing (FDM) first, which is used in FM radio. In FDM the available frequency band is divided into multiple non-overlapping subbands, each of which is used by a different user. Between the subbands are guard bands to prevent interference between the users. This is a very inefficient use of the spectrum.

\begin{figure}[H]
	\centering
	\includegraphics[width=0.4\textwidth]{Figures/fdm.png}
	\caption{Frequency Division Multiplexing}
\end{figure}

OFDM is a more efficient way to divide the spectrum. Instead of using non-overlapping subbands, OFDM divides the spectrum into multiple orthogonal subcarriers. This means that the subcarriers can be spaced closer together, allowing for a higher spectral efficiency. The orthogonality of the subcarriers means that they do not interfere with each other, even when they are spaced closely together. The peaks of the subcarriers line up with the nulls of the other subcarriers, as can be seen in the figure below.

\begin{figure}[H]
	\centering
	\includegraphics[width=0.4\textwidth]{Figures/ofdm.png}
	\caption{Orthogonal Frequency Division Multiplexing}
\end{figure}

\section{MIMO - Freunberger Raphael}
\textbf{MIMO} (or \textbf{m}ultiple-\textbf{i}nput and \textbf{m}ultiple-\textbf{o}utput) is a technology that uses multiple antennas to increase the capacity of a wireless link. 
It is a key technology in modern wireless communication systems, including 4G networks and Wi-Fi. 
MIMO technology takes advantage of \textbf{multi-path propagation}, which means that signals can take multiple paths between the transmitter and receiver caused by reflections, diffractions, and scattering. 
By using multiple antennas at both the transmitter and receiver, MIMO technology can exploit these multiple paths to increase the data rate and reliability of the wireless link.

\begin{figure}[H]
	\centering
	\includegraphics[width=0.4\textwidth]{Figures/mimo.png}
	\caption{Multiple-Input Multiple-Output (MIMO) system} 
\end{figure}

A early predecessor of MIMO technology is \textbf{Space-division multiple access} (SDMA), which used directional antennas to reach multiple users in different directions on the same frequency-band. MIMO technology, on the other hand, uses multiple antennas to transmit multiple data streams on the same frequency-band to the same user.


\subsection{Functions of MIMO}

\paragraph{Channel State Information (CSI)}
Channel State Information (CSI) refers to the knowledge of the communication channel’s characteristics, including attenuation, delay, and fading. 
It helps optimize transmission by adjusting power, phase, and modulation. 
CSI is estimated at the receiver and sent back to the transmitter. 
Accurate CSI is essential for techniques like beamforming and precoding to enhance signal strength and reduce interference.


\paragraph{Precoding and Beamforming}
Precoding is multi-stream beamforming that occurs at the transmitter. 
Beamforming adjusts phase and gain to maximize signal power at the receiver. 
It increases signal strength and reduces multipath fading. 
Precoding with multiple streams is useful when the receiver has multiple antennas and requires CSI.

\paragraph{Spatial Multiplexing}
Spatial multiplexing splits a high-rate signal into lower-rate streams, each transmitted from a different antenna. 
The receiver can separate the streams if it has accurate CSI. 
This technique increases capacity, especially at high SNR. 
The number of streams is limited by the smaller antenna count. 
It can work without CSI at the transmitter and can also support multi-user MIMO with CSI.

\paragraph{Diversity Coding}
Diversity coding transmits a single stream with space-time coding for reliability. 
It exploits independent fading across antennas but does not provide beamforming gains. 
When some CSI is available, diversity coding can combine with spatial multiplexing to improve performance.


\subsection{Variations of MIMO}

Different variations of MIMO systems include:

\begin{itemize}
    \item \textbf{Single-User MIMO (SU-MIMO):} 
    A MIMO system where multiple antennas are used at both the transmitter and receiver to improve data rates and reliability for a single user, using spatial multiplexing to transmit multiple data streams simultaneously.
    
    \item \textbf{Multi-User MIMO (MU-MIMO):} 
    Allows a base station to communicate with multiple users simultaneously, each with multiple antennas. This technique increases system capacity by enabling space-division multiple access (SDMA) based on spatial signatures of the users.
    
    \item \textbf{Massive MIMO:} 
    Involves a large number of antennas at the base station to serve many users simultaneously, greatly enhancing spectral and energy efficiency by exploiting spatial diversity.
    
    \item \textbf{Distributed MIMO:} 
    Utilizes multiple geographically separated antennas connected by a backhaul network, improving coverage and reliability over large areas by exploiting spatial diversity.
    
    \item \textbf{Cooperative MIMO:} 
    Involves collaboration between multiple devices (such as users or base stations) to transmit and receive signals, forming a virtual antenna array. This improves system performance through coordinated resource sharing.
\end{itemize}

\subsection{Antenna Configurations}

The amount of antennas used in a MIMO system on transmitter and receiver is referred to as the antenna configuration. 
Common configurations include 2x2, 4x4, 8x8, etc., where the first number represents the number of antennas on the transmitter and the second number represents the number of antennas on the receiver.
\begin{figure}[H]
	\centering
	\includegraphics[width=0.6\textwidth]{Figures/mimo2x2.png}
	\caption{MIMO 2x2 Uplink and Downlink} 
\end{figure}

\subsubsection{MIMO in 4G Networks}

The 3rd Generation Partnership Project (3GPP) has defined different MIMO configurations for LTE, LTE Advanced, and LTE Advanced Pro networks.

\begin{table}[H]
    \centering
    \begin{tabular}{|l|l|l|}
    \hline
    \textbf{LTE Technology}      & \textbf{3GPP Release} & \textbf{Antenna Configuration (MIMO)} \\ \hline
    LTE                          & Release 8             & 4 x 4 Downlink \newline 2 x 2 Uplink  \\ \hline
    LTE Advanced                 & Release 10            & 8 x 8 Downlink \newline 4 x 4 Uplink  \\ \hline
    LTE Advanced Pro             & Release 13            & 8 x 8 Downlink \newline 4 x 4 Uplink  \\ \hline
    \end{tabular}
    \caption{MIMO configurations in LTE networks}
\end{table}
    

\end{document}