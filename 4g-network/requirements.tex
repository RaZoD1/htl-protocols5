A 4th generation (4G) mobile radio system was designed to fulfill a set of stringent functional and performance requirements, significantly surpassing the capabilities of its 3G predecessors. These requirements were driven by the increasing demand for high-speed data, multimedia services, and seamless mobility. The International Telecommunication Union-Radio communications sector (ITU-R) played a crucial role in defining these requirements.

\subsection{Performance Requirements}

The ITU-R formally defined the performance requirements through the International Mobile Telecommunications Advanced (IMT-Advanced) specification. The requirements were set quite ambitious for the 2008, with peak data rates of 100 megabits per second (Mbit/s) for high mobility communication and 1 gigabit per second (Gbit/s) for low mobility communication. These peak data rates were intended to support a wide range of data-intensive applications, including high-definition video streaming, online gaming, and cloud services.

These requirements could not be met by the first verson of LTE, and ITU-R later droped the ambitious performance requirements. Calling LTE anything other than 4G would have caused significant confusion, as it is not compatible with 3G and a major improvement.

Other major performance requirements outlined in the IMT-Advanced specification include:

\begin{itemize}
    \item \textbf{Spectral Efficiency:} 4G aims for significantly improved spectral efficiency compared to 3G. As the spectrum avalible for civilian use is very limited improved spectral efficiency was required to seve more useres with increasing data rates. Technologies like OFDMA and MIMO were parts of the IMT-Advanced requirements for achieving this for achieving better spectral efficiency. The IMT-Advanced specification targeted peak spectral efficiencies of 15 bps/Hz for the downlink and 6.75 bps/Hz for the uplink.
    \item \textbf{Latency:} Reduced latency is a key requirement for 4G. The latency for establishing a connection with a cell tower should be less than 100ms and the round-trip time for data packets should be less than 10ms.
    \item \textbf{Mobility:} The system is required to support high mobility. The requirements specify that the system must support travel speeds of up to 350km/h.
\end{itemize}

\subsection{Functional Requirements}

Functional requirements define the core capabilities and features that a 4G system must possess. Key functional requirements include:

\begin{itemize}
    \item \textbf{Seamless Mobility and Handover:} 4G systems must provide seamless mobility across different cells and even between different link technologies.
    \item \textbf{Quality of Service (QoS) Differentiation:} 4G networks must be capable of differentiating between different types of traffic and providing varying levels of QoS.
    \item \textbf{Security:} Different combinations of integrety and encryption must be supported to ensure the security of user data and privacy.
\end{itemize}