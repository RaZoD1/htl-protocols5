%%%%%%%%%%%%%%%%%%%%%%%%%%%%%%%%%%%%%%%%%
% Wenneker Assignment
% LaTeX Template
% Version 2.0 (12/1/2019)
%
% This template originates from:
% http://www.LaTeXTemplates.com
%
% Authors:
% Vel (vel@LaTeXTemplates.com)
% Frits Wenneker
%
% License:
% CC BY-NC-SA 3.0 (http://creativecommons.org/licenses/by-nc-sa/3.0/)
% 
%%%%%%%%%%%%%%%%%%%%%%%%%%%%%%%%%%%%%%%%%

%----------------------------------------------------------------------------------------
%	PACKAGES AND OTHER DOCUMENT CONFIGURATIONS
%----------------------------------------------------------------------------------------

\documentclass[11pt]{scrartcl} % Font size

\input{structure.tex} % Include the file specifying the document structure and custom commands

%----------------------------------------------------------------------------------------
%	TITLE SECTION
%----------------------------------------------------------------------------------------

\title{	
	\normalfont\normalsize
	\begin{center}
		\begin{minipage}[c]{0.2\textwidth}
			\textsc{\Large SZ-Ybbs}
		\end{minipage}%
		\begin{minipage}[c]{0.1\textwidth}
			\includegraphics[width=\textwidth]{LogoITHTL_white.pdf}
		\end{minipage}
	\end{center}
	\vspace{10pt} % Whitespace
	\rule{\linewidth}{0.5pt}\\ % Thin top horizontal rule
	\vspace{20pt} % Whitespace
	{\huge TLS}\\ % The assignment title
	\vspace{12pt} % Whitespace
	\rule{\linewidth}{2pt}\\ % Thick bottom horizontal rule
	\vspace{12pt} % Whitespace
}

\author{\LARGE Erber Jakob, Freunberger Raphael} % Your name

\date{\normalsize\today} % Today's date (\today) or a custom date

\begin{document}

\maketitle % Print the title

\section{TLS 1.2/1.3 Overview}

Transport Layer Security (TLS) is a cryptographic protocol used for providing secure communications over an insecure network, like the internet. TSL provides security, confidentiality authenticate and integrity. TSL builds on top of the now-deprecated Secure Sockets Layer (SSL) protocol and does not clearly fit one OSI-layer. Depending on who you ask TLS can be a layer 4, 6 or 7 protocol. The protocol uses TCP, but there is also a UDP equivalent called Datagram Transport Layer Security (DTLS). DTLS is sometimes used for VPNs but in general used less than TSL. TSL has a wide range of use cases, from email to voice over IP or web browsing.
\\\\
To keep its security promise TLS has to keep being developed and new versions get released as the older versions become insecure and get deprecated.

\begin{table}[h]
    \centering
    \begin{tabular}{|l|c|l|}
        \hline
        \textbf{Version} & \textbf{Year} & \textbf{Status} \\ \hline
        TLS 1.0 (insecure) & 1999 & Deprecated in 2021 (RFC 8996) \\ \hline
        TLS 1.1 (insecure) & 2006 & Deprecated in 2021 (RFC 8996) \\ \hline
        TLS 1.2 & 2008 & In use since 2008 \\ \hline
        TLS 1.3 & 2018 & In use since 2018 \\ \hline
    \end{tabular}
    \caption{TLS Versions and Status}
    \label{tab:tls_versions}
\end{table}



\section{Protocols}

\subsection{TLS 1.2}

TLS 1.2 was released in 2008 and is still considered to be secure. Its biggest change compared to TLS 1.1 is the replacement of the MD5 and SHA-1 hash functions with SHA-256 for all security uses. Which is also the reason why TSL 1.2 is still secure and TLS 1.1 is not. However TLS 1.2 still can still use a MD5 or SHA-1 HMAC for data integrity. TLS 1.2 is still widely used and supported by most modern web browsers and servers. The protocol is defined in RFC 5246.
\\\\
In 2011 refinements defined in RFC 6176 removed the backward compatibility with SSL 2.0.

\subsection{TLS 1.3}

\subsubsection{TLS 1.3 Compared to Previous Versions}

\section{Subprotocols}

\section{TLS 1.2 vs TLS 1.3}

\section{TLS 1.2/1.3 Handshake}

\section{Security}

\subsection{Security Features}

\subsection{Attacks}

\section{TLS Record Format}


\end{document}
