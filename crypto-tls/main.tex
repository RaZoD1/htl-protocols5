%%%%%%%%%%%%%%%%%%%%%%%%%%%%%%%%%%%%%%%%%
% Wenneker Assignment
% LaTeX Template
% Version 2.0 (12/1/2019)
%
% This template originates from:
% http://www.LaTeXTemplates.com
%
% Authors:
% Vel (vel@LaTeXTemplates.com)
% Frits Wenneker
%
% License:
% CC BY-NC-SA 3.0 (http://creativecommons.org/licenses/by-nc-sa/3.0/)
% 
%%%%%%%%%%%%%%%%%%%%%%%%%%%%%%%%%%%%%%%%%

%----------------------------------------------------------------------------------------
%	PACKAGES AND OTHER DOCUMENT CONFIGURATIONS
%----------------------------------------------------------------------------------------

\documentclass[11pt]{scrartcl} % Font size

\input{structure.tex} % Include the file specifying the document structure and custom commands

%----------------------------------------------------------------------------------------
%	TITLE SECTION
%----------------------------------------------------------------------------------------

\title{	
	\normalfont\normalsize
	\begin{center}
		\begin{minipage}[c]{0.2\textwidth}
			\textsc{\Large SZ-Ybbs}
		\end{minipage}%
		\begin{minipage}[c]{0.1\textwidth}
			\includegraphics[width=\textwidth]{LogoITHTL_white.pdf}
		\end{minipage}
	\end{center}
	\vspace{10pt} % Whitespace
	\rule{\linewidth}{0.5pt}\\ % Thin top horizontal rule
	\vspace{20pt} % Whitespace
	{\huge TLS}\\ % The assignment title
	\vspace{12pt} % Whitespace
	\rule{\linewidth}{2pt}\\ % Thick bottom horizontal rule
	\vspace{12pt} % Whitespace
}

\author{\LARGE Erber Jakob, Freunberger Raphael} % Your name

\date{\normalsize\today} % Today's date (\today) or a custom date

\begin{document}

\maketitle % Print the title

\section{TLS 1.2/1.3 Overview}

Transport Layer Security (TLS) is a cryptographic protocol used for providing secure communications over an insecure network, like the internet. TSL provides security, confidentiality authenticate and integrity. TSL builds on top of the now-deprecated Secure Sockets Layer (SSL) protocol and does not clearly fit one OSI-layer. Depending on who you ask TLS can be a layer 4, 6 or 7 protocol. The protocol uses TCP, but there is also a UDP equivalent called Datagram Transport Layer Security (DTLS). DTLS is sometimes used for VPNs but in general used less than TSL. TSL has a wide range of use cases, from email to voice over IP or web browsing.
\\\\
To keep its security promise TLS has to keep being developed and new versions get released as the older versions become insecure and get deprecated.

\begin{table}[h]
    \centering
    \begin{tabular}{|l|c|l|}
        \hline
        \textbf{Version} & \textbf{Year} & \textbf{Status} \\ \hline
        TLS 1.0 (insecure) & 1999 & Deprecated in 2021 (RFC 8996) \\ \hline
        TLS 1.1 (insecure) & 2006 & Deprecated in 2021 (RFC 8996) \\ \hline
        TLS 1.2 & 2008 & In use since 2008 \\ \hline
        TLS 1.3 & 2018 & In use since 2018 \\ \hline
    \end{tabular}
    \caption{TLS Versions and Status}
    \label{tab:tls_versions}
\end{table}



\section{Protocols}

\subsection{TLS 1.2}

TLS 1.2 was released in 2008 and is still considered to be secure. Its biggest change compared to TLS 1.1 is the replacement of the MD5 and SHA-1 hash functions with SHA-256 for all security uses. Which is also the reason why TSL 1.2 is still secure and TLS 1.1 is not. However TLS 1.2 can still use a MD5 or SHA-1 HMAC for data integrity. TLS 1.2 is still widely used and supported by most modern web browsers and servers. The protocol is defined in RFC 5246.
\\\\
In 2011 refinements defined in RFC 6176 removed the backward compatibility with SSL 2.0.

\subsection{TLS 1.3}

TLS 1.3 was the largest change in the history of TLS. The TLS 1.3 requirement for forward secrecy meant that a lot of the previously supported key exchange/agreement algorithms were no longer supported. The only key exchange/agreement algorithms supported by TLS 1.3 are different versions of Diffie-Hellman (DH), Elliptic Curve Diffie-Hellman (ECDH) and one algorithm from the Russian GOST standards organization. The supported ciphers were also reduced to only 4 still secure ciphers. Including AES GCM, AES CCM, GOST R34.12 MGM and the stream cipher ChaCha20. The protocol is defined in RFC 8446.
\\\\
The algorithms used for data integrity was reduced to a single one, AEAD.
\\\\
Where as the security of TLS 1.2 depends on which cipher and key exchange method is used, TLS 1.3 only supports secure methods.

\subsubsection{TLS 1.3 Compared to Previous Versions}

\section{Subprotocols}

TLS is a complex protocol that consists of several subprotocols. Each subprotocol has its own purpose and is used in a specific part of TLS. Additionally these subprotocols can be assigned to 2 layers.
\\
The first layer consists of the TLS Record Protocol and is responsible for encapsulating all other subprotocols. The second layer consists of the Handshake Protocol, Alert Protocol, ChangeCipherSpec Protocol and Application Data Protocol. These subprotocols are used to establish a secure connection between the client and the server, and to exchange data securely.


\subsection{TLS Record Protocol}

The TLS Record Protocol is the lowest layer of the TLS protocol stack. It is responsible for encapsulating all other subprotocols. The Record Protocol is used to transmit data securely between the client and the server. The data is end-to-end encrypted, authenticated and integrity protected. The Record Protocol is used to encapsulate the Handshake Protocol, Alert Protocol, ChangeCipherSpec Protocol and Application Data Protocol.

\begin{table}[htbp]
	\centering
	\begin{tabular}{|c|c|c|c|c|}
	\hline
	\textbf{Offset} & \textbf{Byte+0} & \textbf{Byte+1} & \textbf{Byte+2} & \textbf{Byte+3} \\ \hline
	\textbf{Byte 0} & Content type & \multicolumn{3}{c|}{-} \\ \hline
	\multirow{2}{*}{\textbf{Bytes 1-4}} & \multicolumn{2}{c|}{Legacy version} & \multicolumn{2}{c|}{Length} \\ \cline{2-5} 
	 & (Major) & (Minor) & (bits 15-8) & (bits 7-0) \\ \hline
	\textbf{Bytes 5-(\textit{m}-1)} & \multicolumn{4}{c|}{Protocol message(s)} \\ \hline
	\textbf{Bytes \textit{m}-(\textit{p}-1)} & \multicolumn{4}{c|}{MAC (optional)} \\ \hline
	\textbf{Bytes \textit{p}-(\textit{q}-1)} & \multicolumn{4}{c|}{Padding (block ciphers only)} \\ \hline
	\end{tabular}
	\caption{TLS record format, general}
    \label{tab:tls_record_format}
\end{table}
	
The table \ref{tab:tls_record_format} shows the general format of a TLS record. The record consists of:
\begin{itemize}
    \item \textbf{Content type:} The type of the protocol message (ChangeCipherSpec, Alert, Handshake, Application, Heartbeat).
    \item \textbf{Legacy version:} The version of the protocol prior to TLS 1.3. For TLS 1.3, this field must set to 0x0303 and the version is negotiated in the handshake.
    \item \textbf{Length:} The length of the protocol message(s) MAC and padding combined. The maximum length is 2\^14 bytes (16KiB)
    \item \textbf{Protocol message(s):} One or more protocol messages. (may be encrypted)
    \item \textbf{MAC:} The message authentication code. (optional)
    \item \textbf{Padding:} Padding for block ciphers. (block ciphers only)
\end{itemize}


\subsection{TLS Handshake Protocol}

The TLS Handshake Protocol is based on the TLS Record Protocol and is used in most messages of the setup of a TLS session. For a handshake the Content Type field is set to the corresponding value for the Handshake Protocol (0x16) and multiple handshake blocks can be encapsulated in the Protocol message(s) field of the record. 

One handshake block consists of the following fields:
\begin{itemize}
    \item \textbf{Message Type:} the type of the handshake message (HelloRequest, ClientHello, ServerHello, NewSessionTicket, EncryptedExtensions, Certificate, ServerKeyExchange, CertificateRequest, ServerHelloDone, CertificateVerify, ClientKeyExchange, Finished). (1 Byte)
    \item \textbf{Handshake Message Data Length:} the length of the handshake message data. (3 Bytes)
    \item \textbf{Handshake Message Data:} the data of the handshake message. (variable length)
\end{itemize}

Multiple handshake blocks can be sent in one record, which is useful to reduce the number of records sent over the network. 

\subsection{TLS Change Cipher Spec Protocol}

The ChangeCipherSpec Protocol is the simplest of the subprotocols, as it only consists of a single message. The ChangeCipherSpec Protocol is used to signal to the other party that the following records will be encrypted with the CipherSuite negotiated in the handshake. 
The ChangeCipherSpec Protocol contains a 1 byte message with the value 1. This message is sent in the Protocol message(s) field of the record and the Content Type field is set to the corresponding value for the ChangeCipherSpec Protocol (0x14).

The reason ChangeCipherSpec was made into a separate subprotocol was, because it is not wanted for it to be sent alongside other messages in one record.  

\subsection{TLS Alert Protocol}

The TLS Alert Protocol can be used to signal to the other party that a problem occured. It can be send at any time during the connection to notify the other party of the reason of the alert. Usually after an alert is send, the connection is immediately closed or the other party responds with its own alert. However, if the alert is a warning, the connection can be continued if the receiving parties agrees to it. The Alert Protocol is based on the TLS Record Protocol and the Content Type field is set to the corresponding value for the Alert Protocol (0x15) and wraps the alert message in the Protocol message(s) field of the record.

The Alert Protocol consists of the following fields:
\begin{itemize}
    \item \textbf{Level:} The level of the alert (warning or fatal). (1 Byte)
    \item \textbf{Description:} defines which type of alert is sent (1 Byte)
\end{itemize}

There are several codes for the Description field, to specify the cause of the alert. For example Handshake Failure, Bad Certificate, Unkown CA, etc.

\subsection{TLS Application Data Protocol}

The TLS Application Data Protocol is used to transmit data of an underlying application. The Application Data Protocol is based on the TLS Record Protocol and the Content Type field is set to the corresponding value for the Application Data Protocol (0x17) and the application data is sent in the Protocol message(s) field of the record. The Application Data Protocol is used to transmit the actual data of the application, like the content of a website, a file or a message.
Additonally the MAC and padding may be included in the record, depending on the cipher used.


\section{TLS 1.2 vs TLS 1.3}

\section{TLS 1.2/1.3 Handshake}

\section{Security}

\subsection{Security Features}

\subsection{Attacks}

\section{TLS Record Format}


\begin{table}[htbp]
	\centering
	\caption{TLS record format, general}
	\begin{tabular}{|c|c|c|c|c|}
	\hline
	\textbf{Offset} & \textbf{Byte+0} & \textbf{Byte+1} & \textbf{Byte+2} & \textbf{Byte+3} \\ \hline
	\textbf{Byte 0} & Content type & \multicolumn{3}{c|}{-} \\ \hline
	\multirow{2}{*}{\textbf{Bytes 1-4}} & \multicolumn{2}{c|}{Legacy version} & \multicolumn{2}{c|}{Length} \\ \cline{2-5} 
	 & (Major) & (Minor) & (bits 15-8) & (bits 7-0) \\ \hline
	\textbf{Bytes 5-(\textit{m}-1)} & \multicolumn{4}{c|}{Protocol message(s)} \\ \hline
	\textbf{Bytes \textit{m}-(\textit{p}-1)} & \multicolumn{4}{c|}{MAC (optional)} \\ \hline
	\textbf{Bytes \textit{p}-(\textit{q}-1)} & \multicolumn{4}{c|}{Padding (block ciphers only)} \\ \hline
	\end{tabular}
	\end{table}
	
	
	

\end{document}
